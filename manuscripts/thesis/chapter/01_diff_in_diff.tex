\chapter{Overview of DiD with Staggered Treatment from Callaway and Sant'Anna (2021)}
\label{sec:overview_did}

In recent years, numerous authors have advanced the Difference-in-Differences
design literature by addressing the ``forbidden comparisons'' issue, term coined and explained in
\textcite[14]{borusyak.etal2024revisiting}. In short, it arises when comparing later treated to early
treated units in a staggered DiD setting in the presence of effect heterogeneity. One of these approaches was proposed by
\textcite{cs2021did_mtp}, which is not only robust to this issue but offers other compelling
features, as discussed below. For an overview on the recent methods refer to the extensive survey from
\textcite{dechaisemartin.dhaultfoeuille2022twoway}.

In the following section I present an overview of the estimator as proposed from
\textcite{cs2021did_mtp}. I include only the sections that are relevant to this analysis,
while variations and other considerations are left out. For consistency, I borrow the terminology from the authors
for this analysis. In particular, I use terms such as ``comparison'' units instead of commonly used
``control'' units. I also use the term ``group'' which can be exchanged for ``cohort''.


\section{Building Blocks}

In short, the proposed estimator can be divided into two steps. First, compute the differences of valid
comparison groups in several simple, balanced 2-periods by 2-groups DiD designs. Then, aggregate
these estimations in a meaningful way. Adapting the notation of Eq. 2.8 from
\parencite{cs2021did_mtp}, the groups in the first step can be defined by their respective treatment
period $g$ and the measurement time $t$. So each group-time Average Treatment Effects on the Treated, 
hereafter $ATT(g,t)$ is defined as%
%
\begin{equation} 
    ATT(g,t) = \E \big[ Y_{t} - Y_{g-1} | G_{g}=1 \big]  -  \E \big[ Y_{t} - Y_{g-1} |
    C=1 \big], 
    \label{eq:attgt}
\end{equation}
%
where the conditional terms $G_g=1$ identifies the group first treated at period $g$ and $C=1$
determines the comparison units.%
\footnote{Note that the authors introduce two approaches that differs in terms of comparison units.
    One uses never-treated only, while the other uses also not-yet-treated units as comparison. I focus
    here on the never-treated specification, which is used in the main analysis, due to the risk of
    anticipation effects biasing the results.} %
%
This is the \textit{group-time average treatment effect}, as defined by the authors
\parencite*[Section 2.2]{cs2021did_mtp}. Since each one of these groups is by construction, a
canonical DiD design, they are valid comparisons and do not suffer from the ``negative weights''
issue.

In the second step, the \textcite{cs2021did_mtp} propose different ways of aggregating all $ATT(g,t)$'s.
In the main analysis, I follow the event study aggregation. For that, a relative-time 
variable $e = t-g$ is defined, which identifies time elapsed since treatment adoption. The target 
parameter is
%%
%
\begin{equation}
    \theta_\text{es}(e) = \sum_{g \in \mathcal{G}} \mathbb{1}\big[ g + e \leq \mathcal{T} \big] P(G=g|G+e \leq \mathcal{T}) ATT(g,g+e), 
    \label{eq:theta_es}
\end{equation}
%
where $\mathcal{T}$ is the last period in the analysis. The two new terms in the above summation are, first,
the indicator function that restricts the group-time average treatment effects. And second, the term $P(.)$ are
the group weights, so that the summation produces an average - weighted by group size - of all $ATT(g,t)$'s
included in the aggregation.


Note that this aggregation may suffer from compositional changes, which will have important implications, as
explained thoroughly in \textcite[see Equation 3.5]{cs2021did_mtp}. In short, the authors show that there are
two approaches to handling it: either restricting the analysis to a balanced sample or, alternatively, relying
on the additional assumption that the effects are equal across groups. That is, $ATT(g,g+e)$ does not vary with
$g$ for any $e \geq 0$. In the main analysis, I make the equal-effects assumption, but robustness checks with
sample restrictions do not change the results considerably.

\section{Summary Measures}

After computing the $ATT(g,t)$'s, there are a few choices of overall aggregation to obtain a single
parameter estimating the effect being assigned to treatment. As proposed by \textcite[Section
3.2]{cs2021did_mtp}, an intuitive one would be

\begin{equation}
    \overbar{\theta}_\text{simple} = \frac{1}{k} \sum_{g \in \mathcal{G}}^{} 
    \sum_{t=2}^{\mathcal{T}} \mathbb{1}\big[ t \geq g \big] ATT(g,t) P(G=g|G \leq \mathcal{T}), 
    \label{eq:didsimp}
\end{equation}
%
where $k=\sum_{g \in \mathcal{G}} \sum_{t=2}^{\mathcal{T}} \mathbb{1}\big[ t \geq g \big]
ATT(g,t) P(G=g|G \leq \mathcal{T})$. Another aggregation target proposed is relevant for the
event-study structure applied here. We can obtain an overall treatment effect by averaging
$\theta_{es}(e)$ over all event times, as in%
%
\begin{equation}
    \overbar{\theta}_\text{es} = \frac{1}{\mathcal{T-1}} \sum_{e=0}^{\mathcal{T}-2}{\theta_{es}(e)}.
    \label{eq:didpostpast}
\end{equation}

While the authors propose other summary measures, these are the relevant ones examined in this
analysis and presented in \cref{sec:Results}. %todo: cross ref main results

\section{Doubly-Robustness}

\citeauthor{cs2021did_mtp} propose three approaches to achieve an estimator capable of conducting
inference on the group-time average treatment effects \autocite[Sec. 4]{cs2021did_mtp}. In short, the Outcome
Regression, a generalization of the approach in \textcite{heckman.etal1997matching}, requires the evolution of
the response variable of the comparison group to be modeled correctly to ensure conditional parallel trends.
The second approach, building on inverse probability weighting (IPW) proposed by
\textcite{abadie2005semiparametric}, requires correct estimation of the conditional probability of being in
group $g$.


The doubly-robust method, extending the simpler DiD estimator from
\textcite{santanna.zhao2020doubly} to allow for multiple \textit{group-times}, is particularly appealing
because it can be seen as a combination of both approaches, modeling the response variable evolution
as well as the propensity score, but it only requires one to be correctly specified. Thus the 
term ``doubly robust''. 


For a more detailed view on the proposed estimator, see \cref{app:drdeep}. A closer look at
\cref{eq:wgtreatcomp} shows that the treated group are not re-weighted (or all receive unit
weights). Further, the subscripts in $\widehat{ATT}_{d r}(g, t)$, in \cref{eq:attdr}, hints that
this procedure is conducted for each one of the 2-by-2 group-time $ATT$, which entails important
implications: the same comparison units, those that are never treated in this case, are used for all
group-time estimations but receive different weights in each \textit{step}.


The implication is that it is not trivial to extract and store the weights from the IPW procedure.%
\footnote{The main author of the \code{csdid} package for Stata \parencite{rios-avila.etal2023csdid}
    confirms this in the Statalist forum \parencite{rios-avila2023SLWeights}. One could extract the
    weights from each treatment time cohort, and use them for further checks. In a setting with
    more than a few cohorts, however, this quickly becomes infeasible.} This means that while
this procedure doubly robust and automatically ``balances'' treatment and comparison groups, the
researcher cannot access the same weights to double check if the estimator is doing a good job at
that. This is a considerable drawback, because it reduces the ability of the research to better
explore some of the identifying assumptions of causality.

% -------------------------------------------------------------------------------------------------
% Key Assumptions of DiD design ----
% -------------------------------------------------------------------------------------------------
%todo: lead to next subsection
\section{Key Identifying Assumption}

As in conventional DiD settings, the key identifying assumption is of common trends. That is, the
response variable of the treated units would follow the same trend of that from comparison units
also in the absence of treatment. If the assumption holds, we have a credible counterfactual 
with which we can estimate causal treatment effects. Since this assumption is not testable, we
refer to pre-treatment trends to assess its plausibility. If, prior to treatment, the eventually
treated and comparison units follow a similar trend, it is reasonable to infer that the 
trend would have remained parallel after treatment time also in the \textit{counterfactual world} where 
the in the treated units were not treated. 

It is also important to note that while pre-treatment balance can boost the plausibility of the
parallel trends assumption, balanced groups are not actually a requirement. This strategy relies on
the assumption that solely the trends are parallel, that is, the evolution of the response variable
are the same across groups, regardless of starting values.

The pre-treatment periods can also be useful to assess the presence of reverse-causality. In this
setting, reverse causality would imply that wealth outcomes (or lack thereof) in early periods
actually causing the steeper health degradation, which ultimately causes treatment assignment
eventually. Given the nature of these two variables at hand, how they might, a priori, interact and
slowly evolve over the life time, this is a serious reason of concern.





