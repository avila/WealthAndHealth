% -------------------------------------------------------------------------------------------------
% input figures with histograms of mental and physical health questions ----
% -------------------------------------------------------------------------------------------------




\newcommand{\vlgwnlog}{Gross Wealth (log)}                   % gw_nlog
\newcommand{\vlnwnlog}{Net Wealth (neglog)}                  % nw_nlog
\newcommand{\vlgw}{Gross Wealth (k€, winsored)}              % gw
\newcommand{\vlnw}{Net Wealth (k€, winsored)}                % nw
\newcommand{\vlexpue}{Unemployment exp. (months)}            % expue
\newcommand{\vlexpft}{Full-Time exp. (months)}               % expft
\newcommand{\vlemplstatus}{Employment Status of Individual}  % empl_status
\newcommand{\vllaborearns}{Individual Labor Earnings}        % labor_earns
\newcommand{\vlcardioev}{Cardiopathy}                        % cardio_ev
\newcommand{\vldepresev}{Depression}                         % depres_ev


% -------------------------------------------------------------------------------------------------
%  {\vlgwnlog}{Gross Wealth (log)}                   % gw_nlog
% -------------------------------------------------------------------------------------------------
% -------------------------------------------------------------------------------------------------
\begin{figure}[htb!]
    \centering
    \begin{subfigure}{0.45\textwidth}
        \caption{Gross Wealth (log)\qquad Models: P1\tsub{a} / M1\tsub{a}}
        \includegraphics[width=.95\linewidth]{../../output/figures/csdid2/f_robust/ep1_gw_nlog.pdf}
        \label{sfig:ep1_gw_nlog}
    \end{subfigure}
    \begin{subfigure}{0.45\textwidth}
        \caption{Gross Wealth (log)\qquad Models: P1\tsub{b} / M1\tsub{b}}
        \includegraphics[width=.95\linewidth]{../../output/figures/csdid2/f_robust/ep2_gw_nlog.pdf}
        \label{sfig:ep2_gw_nlog}
    \end{subfigure}
    \begin{subfigure}{0.45\textwidth}
        \caption{Gross Wealth (log)\qquad Models: P1\tsub{c} / M1\tsub{c}}
        \includegraphics[width=.95\linewidth]{../../output/figures/csdid2/f_robust/ep3_gw_nlog.pdf}
        \label{sfig:ep3_gw_nlog}
    \end{subfigure}
    \begin{subfigure}{0.45\textwidth}
        \caption{Gross Wealth (log)\qquad Models: P1\tsub{d} / M1\tsub{d}}
        \includegraphics[width=.95\linewidth]{../../output/figures/csdid2/f_robust/ep4_gw_nlog.pdf}
        \label{sfig:ep4_gw_nlog}
    \end{subfigure}
    \caption{Varying model specification}
    \label{fig:ep_gw_nlog}
    \fnote{Notes: Replications with $\log(Gross~Wealth)$ as response variable with  different specifications to check for model dependence.\\
        Models P1\tsub{a} and M1\tsub{a} include not-yet-treated (as well as never-treated) as comparison units. \\ 
        Models P1\tsub{b} and M1\tsub{b} do not include any covariates. \\
        Models P1\tsub{c} and M1\tsub{c} include covariates and are weighted by the inverse probability of remaining in the SOEP, to account for sample attrition might have differential impact over the wealth
        distribution. \\
        Models P1\tsub{d} and M1\tsub{d} are balanced, estimated with people present from 2002 to 2020.
        %                todo: this part to main text body 
        The coefficients are stable across models and similar to those obtained from the 
        main specifications P1 and M1. The estimated coefficients of the mental
        health domain are consistently higher than those of the physical domain. 
        Coefficients (and standard errors) already transformed to represent the effect in percentage terms. 
        The same results are also presented in \cref{tab:t_cmp_gw_nlog}.}
\end{figure}

%S[table-format=1.3, table-space-text-post ={***}]
% -------------------------------------------------------------------------------------------------
%\newcolumntype{p}{>{\columncolor{cp}}S}
%\newcolumntype{P}{>{}S[table-number-alignment = left]}
%\newcolumntype{P}{>{\columncolor{cp}}S[table-number-alignment = left]}
%\newcolumntype{M}{>{\columncolor{cm}}S[table-number-alignment = left]}
\begin{table}[htb!]
    \begin{adjustbox}{max width=\textwidth}
        \begin{threeparttable}
            \scriptsize
            \caption{Varying model specification}
            \label{tab:t_cmp_gw_nlog}
            %            \setlength{\tabcolsep}{12pt}
            \begin{tabular}{l*{4}{P}*{4}{M}}
                \toprule
                & \multicolumn{4}{l}{Physical Health} & \multicolumn{4}{l}{Mental Health}                             \\ %\cmidrule(lr){2-5} \cmidrule(lr){6-9} 
                & \multicolumn{4}{l}{Gross Wealth (log)} & \multicolumn{4}{l}{Gross Wealth (log)}                             \\ \cmidrule(lr){2-5} \cmidrule(lr){6-9} 
                & {(P1\tsub{a})}       & {(P1\tsub{b})}       & {(P1\tsub{c})}        & {(P1\tsub{d})}        & {(M1\tsub{a})}       & {(M1\tsub{b})}       & {(M1\tsub{c})}        & {(M1\tsub{d})}       \\
                & (\%)       & (\%)               & (\%)                & (\%)                & (\%)       & (\%)               & (\%)                &  (\%)              \\
                \midrule%
                
SimpleATT           &      -0.079\sym{**} &      -0.073\sym{**} &      -0.081\sym{***}&      -0.106\sym{*}  &      -0.085\sym{***}&      -0.081\sym{***}&      -0.095\sym{***}&      -0.111\sym{**} \\
                    &      (0.02)         &      (0.02)         &      (0.02)         &      (0.04)         &      (0.02)         &      (0.02)         &      (0.02)         &      (0.04)         \\
Pre\_avg             &       0.026         &       0.031         &       0.009         &       0.048         &       0.020         &       0.005         &      -0.010         &       0.007         \\
                    &      (0.04)         &      (0.04)         &      (0.04)         &      (0.05)         &      (0.03)         &      (0.03)         &      (0.04)         &      (0.05)         \\
Post\_avg            &      -0.090\sym{**} &      -0.089\sym{**} &      -0.099\sym{**} &      -0.111\sym{**} &      -0.107\sym{***}&      -0.101\sym{***}&      -0.116\sym{***}&      -0.113\sym{**} \\
                    &      (0.03)         &      (0.03)         &      (0.03)         &      (0.04)         &      (0.03)         &      (0.03)         &      (0.03)         &      (0.04)         \\
tm10                &      -0.012         &       0.013         &      -0.014         &       0.040         &       0.034         &       0.020         &      -0.038         &       0.111         \\
                    &      (0.09)         &      (0.09)         &      (0.09)         &      (0.10)         &      (0.07)         &      (0.07)         &      (0.08)         &      (0.09)         \\
tm8                 &       0.068         &       0.073         &       0.033         &       0.074         &       0.004         &      -0.025         &      -0.030         &      -0.048         \\
                    &      (0.05)         &      (0.05)         &      (0.06)         &      (0.07)         &      (0.05)         &      (0.05)         &      (0.05)         &      (0.07)         \\
tm6                 &       0.013         &       0.007         &      -0.016         &       0.032         &       0.010         &      -0.002         &       0.004         &      -0.036         \\
                    &      (0.04)         &      (0.04)         &      (0.04)         &      (0.05)         &      (0.03)         &      (0.03)         &      (0.04)         &      (0.05)         \\
tm4                 &       0.034\sym{*}  &       0.031\sym{*}  &       0.032\sym{*}  &       0.044         &       0.031\sym{*}  &       0.028         &       0.025         &       0.000         \\
                    &      (0.02)         &      (0.02)         &      (0.02)         &      (0.02)         &      (0.02)         &      (0.01)         &      (0.02)         &      (0.03)         \\
tp0                 &      -0.022         &      -0.014         &      -0.021         &      -0.033         &      -0.011         &      -0.010         &      -0.022         &      -0.028         \\
                    &      (0.01)         &      (0.01)         &      (0.01)         &      (0.02)         &      (0.01)         &      (0.01)         &      (0.01)         &      (0.02)         \\
tp2                 &      -0.038\sym{*}  &      -0.026         &      -0.036         &      -0.073\sym{*}  &      -0.032         &      -0.031         &      -0.046\sym{*}  &      -0.049         \\
                    &      (0.02)         &      (0.02)         &      (0.02)         &      (0.03)         &      (0.02)         &      (0.02)         &      (0.02)         &      (0.03)         \\
tp4                 &      -0.099\sym{***}&      -0.095\sym{***}&      -0.099\sym{***}&      -0.099\sym{*}  &      -0.102\sym{***}&      -0.103\sym{***}&      -0.118\sym{***}&      -0.090\sym{*}  \\
                    &      (0.03)         &      (0.03)         &      (0.03)         &      (0.04)         &      (0.03)         &      (0.02)         &      (0.03)         &      (0.04)         \\
tp6                 &      -0.133\sym{***}&      -0.118\sym{***}&      -0.126\sym{***}&      -0.133\sym{**} &      -0.122\sym{***}&      -0.114\sym{***}&      -0.130\sym{***}&      -0.120\sym{**} \\
                    &      (0.04)         &      (0.03)         &      (0.04)         &      (0.05)         &      (0.03)         &      (0.03)         &      (0.03)         &      (0.05)         \\
tp8                 &      -0.148\sym{***}&      -0.136\sym{**} &      -0.147\sym{**} &      -0.136\sym{*}  &      -0.169\sym{***}&      -0.150\sym{***}&      -0.167\sym{***}&      -0.166\sym{**} \\
                    &      (0.04)         &      (0.04)         &      (0.05)         &      (0.06)         &      (0.04)         &      (0.04)         &      (0.04)         &      (0.05)         \\
tp10                &      -0.111\sym{*}  &      -0.131\sym{**} &      -0.148\sym{**} &      -0.177\sym{**} &      -0.125\sym{**} &      -0.122\sym{**} &      -0.139\sym{**} &      -0.161\sym{**} \\
                    &      (0.05)         &      (0.05)         &      (0.05)         &      (0.06)         &      (0.05)         &      (0.04)         &      (0.05)         &      (0.06)         \\
tp12                &      -0.081         &      -0.103         &      -0.114         &      -0.127         &      -0.187\sym{**} &      -0.176\sym{***}&      -0.188\sym{***}&      -0.178\sym{**} \\
                    &      (0.07)         &      (0.06)         &      (0.07)         &      (0.07)         &      (0.06)         &      (0.05)         &      (0.06)         &      (0.06)         \\
\midrule
Pretrend $\chi^2$ (df)&       26.71         &       28.00         &       22.61         &       36.19         &       16.45         &       19.18         &       20.14         &       65.88         \\
Pretrend p-value    &  22 (0.222)         &  22 (0.176)         &  22 (0.424)         &  28 (0.138)         &  22 (0.793)         &  22 (0.634)         &  22 (0.574)         &  28 (0.000)         \\
Covariates          &                     &$\checkmark$         &$\checkmark$         &$\checkmark$         &                     &$\checkmark$         &$\checkmark$         &$\checkmark$         \\
Inv.Pr(stay)        &                     &                     &$\checkmark$         &                     &                     &                     &$\checkmark$         &                     \\
Balanced            &                     &                     &                     &$\checkmark$         &                     &                     &                     &$\checkmark$         \\
\bottomrule

            \end{tabular}
            \begin{tablenotes}[para,flushleft]
                \vspace*{-\baselineskip} {\raggedleft*~$p<0.05$,~**~$p<0.01$,~***~$p<0.001$\\}
                Notes: Replications with $\log(Gross~Wealth)$ as response variable with different specifications to check for model dependence.  
                Wild Bootstrap standard error in parenthesis. 
                \item Models P1\tsub{a} and M1\tsub{a} include not-yet-treated (as well as never-treated) as comparison units. \\ 
                \item Models P1\tsub{b} and M1\tsub{b} do not include any covariates. \\
                \item Models P1\tsub{c} and M1\tsub{c} include covariates and are weighted by the inverse probability of remaining in the SOEP, to account for sample attrition might have differential impact over the wealth
                distribution. \\
                \item Models P1\tsub{d} and M1\tsub{d} are balanced, estimated with people present from 2002 to 2020.
                %                todo: this part to main text body 
                The coefficients are stable across models and similar to those obtained from the 
                main specifications P1 and M1. 
                The covariates included are gender, age spline, federal state residence, legal disability, marital status, and years of education.
                Coefficients (and standard errors) already transformed to represent the effect in percentage terms. 
                For a visual representation of the same results see \cref{fig:ep_gw_nlog}.
            \end{tablenotes}
        \end{threeparttable}
    \end{adjustbox}
\end{table} % todo: include number in balanced model
% -------------------------------------------------------------------------------------------------
% -------------------------------------------------------------------------------------------------



% -------------------------------------------------------------------------------------------------
% tab:diff_treat_rule ----
% -------------------------------------------------------------------------------------------------
\begin{table}[htbp!]
    \centering \
    \begin{adjustbox}{max width=1\textwidth}
        \begin{threeparttable}
            \scriptsize
            \caption{Table of results with different treatment rule}
            \label{tab:diff_treat_rule}
            \setlength{\tabcolsep}{6pt}
            \begin{tabular}{l PPPP MMMM}
                \toprule
                & \multicolumn{4}{l}{Physical Health} & \multicolumn{4}{l}{Mental Health}                             \\ \cmidrule(lr){2-5} \cmidrule(lr){6-9} 
                & \multicolumn{2}{l}{(neg)log} & \multicolumn{2}{l}{level} & \multicolumn{2}{l}{(neg)log} & \multicolumn{2}{l}{level} \\ \cmidrule(lr){2-3} \cmidrule(lr){4-5} \cmidrule(lr){6-7} \cmidrule(lr){8-9} 
                & {gross (\%)} & {net (\%)$^1$} & {gross} & {net} & {gross (\%)} & {net (\%)$^1$} & {gross} & {net}           \\ 
                & {(P1\tsub{r})}       & {(P2\tsub{r})}       & {(P3\tsub{r})}        & {(P4\tsub{r})}        & {(M1\tsub{r})}       & {(M2\tsub{r})}       & {(M3\tsub{r})}        & {(M4\tsub{r})}       \\
                \midrule%
                
SimpleATT           &       -3.44\sym{*}  &       -2.77         &       -4.07\sym{*}  &       -4.47\sym{**} &       -7.36\sym{***}&       -8.40\sym{***}&       -4.52\sym{*}  &       -3.97\sym{*}  \\
                    &      (1.70)         &      (2.39)         &      (1.75)         &      (1.56)         &      (1.62)         &      (2.13)         &      (1.80)         &      (1.62)         \\
Pre average             &        1.60         &        4.74         &        3.47         &        2.86         &        3.00         &        4.75         &        1.53         &        2.16         \\
                    &      (2.24)         &      (3.41)         &      (2.12)         &      (1.92)         &      (2.42)         &      (3.12)         &      (2.73)         &      (2.26)         \\
Post average            &       -4.80         &       -3.25         &       -6.47\sym{*}  &       -6.75\sym{**} &       -9.66\sym{***}&      -10.74\sym{***}&       -6.37\sym{*}  &       -5.39\sym{*}  \\
                    &      (2.49)         &      (3.63)         &      (2.70)         &      (2.44)         &      (2.17)         &      (2.91)         &      (2.83)         &      (2.31)         \\
$\hat{\theta}_{es}(-10)$                &        6.04         &        9.38         &        5.62         &        5.11         &        3.22         &        5.53         &        1.15         &        3.89         \\
                    &      (4.67)         &      (7.28)         &      (4.00)         &      (3.72)         &      (5.05)         &      (6.71)         &      (5.40)         &      (4.76)         \\
$\hat{\theta}_{es}(-8)$                 &       -1.76         &        1.73         &        2.80         &        1.86         &        4.50         &        7.53         &        1.98         &        3.48         \\
                    &      (3.27)         &      (4.85)         &      (3.20)         &      (2.86)         &      (3.47)         &      (4.81)         &      (3.54)         &      (3.12)         \\
$\hat{\theta}_{es}(-6)$                 &        1.22         &        4.12         &        4.64\sym{*}  &        3.57\sym{*}  &        2.41         &        1.66         &        1.46         &        0.19         \\
                    &      (2.09)         &      (3.04)         &      (1.97)         &      (1.71)         &      (2.10)         &      (2.89)         &      (2.22)         &      (2.03)         \\
$\hat{\theta}_{es}(-4)$                 &        1.03         &        3.86\sym{*}  &        0.81         &        0.88         &        1.90         &        4.37\sym{**} &        1.54         &        1.09         \\
                    &      (1.12)         &      (1.63)         &      (0.95)         &      (0.86)         &      (1.13)         &      (1.50)         &      (0.97)         &      (0.86)         \\
$\hat{\theta}_{es}(0)$                 &       -0.69         &       -2.45         &       -0.77         &       -1.19         &       -2.41\sym{**} &       -2.76\sym{*}  &       -1.79\sym{*}  &       -1.36\sym{*}  \\
                    &      (0.77)         &      (1.34)         &      (0.74)         &      (0.73)         &      (0.80)         &      (1.16)         &      (0.71)         &      (0.69)         \\
$\hat{\theta}_{es}(2)$                 &       -2.56         &       -3.19         &       -3.42\sym{*}  &       -3.46\sym{*}  &       -4.14\sym{**} &       -5.10\sym{*}  &       -2.53         &       -2.49         \\
                    &      (1.58)         &      (2.50)         &      (1.51)         &      (1.39)         &      (1.53)         &      (2.28)         &      (1.51)         &      (1.42)         \\
$\hat{\theta}_{es}(4)$                 &       -3.94         &       -1.41         &       -4.77\sym{*}  &       -6.06\sym{**} &      -10.91\sym{***}&      -12.36\sym{***}&       -5.25\sym{*}  &       -5.30\sym{**} \\
                    &      (2.26)         &      (3.73)         &      (2.35)         &      (2.17)         &      (2.07)         &      (2.96)         &      (2.37)         &      (2.02)         \\
$\hat{\theta}_{es}(6)$                 &       -6.21\sym{*}  &       -3.57         &       -4.72         &       -5.79\sym{*}  &      -11.42\sym{***}&      -14.55\sym{***}&       -7.17\sym{*}  &       -6.24\sym{*}  \\
                    &      (2.79)         &      (4.39)         &      (2.93)         &      (2.51)         &      (2.46)         &      (3.35)         &      (3.01)         &      (2.78)         \\
$\hat{\theta}_{es}(8)$                 &       -7.76\sym{*}  &       -2.10         &       -6.49         &       -5.63         &      -13.42\sym{***}&      -14.26\sym{**} &       -7.64\sym{*}  &       -6.91\sym{*}  \\
                    &      (3.55)         &      (5.35)         &      (3.87)         &      (3.50)         &      (3.08)         &      (4.23)         &      (3.89)         &      (3.39)         \\
$\hat{\theta}_{es}(10)$                &       -5.71         &       -3.02         &      -10.44\sym{*}  &      -10.35\sym{*}  &      -12.58\sym{**} &      -11.91\sym{*}  &       -7.71         &       -6.59         \\
                    &      (4.18)         &      (6.35)         &      (5.29)         &      (4.45)         &      (3.95)         &      (4.91)         &      (5.20)         &      (4.63)         \\
$\hat{\theta}_{es}(12)$                &       -6.51         &       -6.88         &      -14.66\sym{*}  &      -14.77\sym{*}  &      -12.13\sym{*}  &      -13.50\sym{*}  &      -12.48         &       -8.83         \\
                    &      (6.02)         &      (7.54)         &      (6.82)         &      (5.94)         &      (4.74)         &      (6.13)         &      (6.52)         &      (5.44)         \\
\midrule
Obs                 &       90207         &       90207         &       90207         &       90207         &       84925         &       84925         &       84925         &       84925         \\
Pretrend $\chi^2$ (df)&  24.83 (26)         &  18.10 (26)         &  36.22 (26)         &  38.92 (26)         &  23.44 (26)         &  29.88 (26)         &  36.60 (26)         &  31.64 (26)         \\
Pretrend p-value    &       0.529         &       0.872         &       0.088         &       0.050         &       0.608         &       0.272         &       0.081         &       0.205         \\
\bottomrule

            \end{tabular}
            \begin{tablenotes}[para,flushleft]
                \vspace*{-\baselineskip} 
                {\raggedleft*~$p<0.05$,~**~$p<0.01$,~***~$p<0.001$ \\}
                Notes: Results of replicating the primary model with a different assignment rule. 
                Instead of having to experience a negative health outcome at least twice, people experiencing it 
                only once is already assigned to treatment. Further, the threshold is 1~Std.~Dev. instead of $1/2$ used in the main analysis. 
                In general, the results are similar, but consistently smaller than in the main analysis. The difference is
                around half to two thirds of estimations in the main results. The only model which is not strongly affected
                by these parameter changes is the one with gross wealth in logarithms as the response variable, M1\tsub{r} in the
                robustness checks and M1 in the main analysis. 
                The covariates included are gender, age spline, federal state residence, legal disability, marital status, and years of education.
                $^1$Coefficients (and standard errors) of log specifications are transformed to represent the effect in percentage terms, but
                such interpretation of the neglog transformation might be biased (see \cref{sec:transformcoefs})
                Wild Bootstrap standard error in parenthesis. 
                The main results are presented in \cref{tab:main_res_event} and the corresponding notes 
                on the procedure apply to this table equally. 
            \end{tablenotes}
        \end{threeparttable}
    \end{adjustbox}
\end{table}
% -------------------------------------------------------------------------------------------------
% -------------------------------------------------------------------------------------------------





% -------------------------------------------------------------------------------------------------
% figure: robustness checks (diff treat rule) ----
% -------------------------------------------------------------------------------------------------
\begin{figure}[ht!]
    \centering
    \begin{subfigure}{0.45\textwidth}
        \caption{Gross Wealth (log)\\(P1\tsub{r} / M1\tsub{r})}
        \includegraphics[width=.95\linewidth]{../../output/figures/csdid2/b_mcspcs/f_11_gw_nlog_1_ct1.pdf}
        \label{sfig:did_gw_nlog1ct1}
    \end{subfigure}
    \begin{subfigure}{0.45\textwidth}
        \caption{Net Wealth (neglog)\\(P2\tsub{r} / M2\tsub{r})}
        \includegraphics[width=.95\linewidth]{../../output/figures/csdid2/b_mcspcs/f_21_nw_nlog_1_ct1.pdf}
        \label{sfig:did_nw_nlog1ct1}
    \end{subfigure}
    \begin{subfigure}{0.45\textwidth}
        \caption{Gross Wealth (k€, winsored)\\(P3\tsub{r} / M3\tsub{r})}
        \includegraphics[width=.95\linewidth]{../../output/figures/csdid2/b_mcspcs/f_31_gw_1_ct1.pdf}
        \label{sfig:did_gw1ct1}
    \end{subfigure}
    \begin{subfigure}{0.45\textwidth}
        \caption{Net Wealth (k€, winsored)\\(P4\tsub{r} / M4\tsub{r})}
        \includegraphics[width=.95\linewidth]{../../output/figures/csdid2/b_mcspcs/f_41_nw_1_ct1.pdf}
        \label{sfig:did_nw1ct1}
    \end{subfigure}
    \caption{Event-study results with different treatment assignment rule}
    \label{fig:diff_treat_rule}
    \fnote{Notes: The results corresponding to this panel is presented in \cref{tab:diff_treat_rule}.
        The same notes apply to this figure. }
\end{figure}



% pcs_main        // Physical Health (oblique)
% gw_nlog         // Gross Wealth (log)
% gw              // Gross Wealth (k€, winsored)
% pcs_main        // Physical Health (oblique)
% empl_status     // Employment Status of Individual
% sats_health     // Satisfaction With Health
% mcs_main        // Mental Health (oblique)
% gw_nlog         // Gross Wealth (log)
% gw              // Gross Wealth (k€, winsored)
% mcs_main        // Mental Health (oblique)
% expue           // Unemployment exp. (months)
% sats_health     // Satisfaction With Health




\begin{figure}[htb!]
    % -------------------------------------------------------------------------------------------------
    % var: pcs_main         % Physical Health Domain
    % -------------------------------------------------------------------------------------------------
    % -------------------------------------------------------------------------------------------------
    \centering \setcounter{subfigure}{0}% Reset subfigure counter
    %%A: Physical Health Domain
    %\\ %%%%%%%%%%%%%%%%%%%%%%%%%%%%%%%%%%%%%%%%%%%%%%%%%%%%%%%%%%%%%%%%%%%%%%%%%%%%%%%%%%%%%%%%%%%
    \begin{subfigure}{0.32\textwidth}
        \caption{Physical Health (oblique)}
        \includegraphics[width=.95\linewidth]{../../output/figures/csdid2/h_descr/f_pcs_main_pcs_main_n1_win1.pdf}
        \label{sfig:fpcsmainpcsmain}
    \end{subfigure}
    \begin{subfigure}{0.32\textwidth}
        \caption{Gross Wealth (log)}
        \includegraphics[width=.95\linewidth]{../../output/figures/csdid2/h_descr/f_gw_nlog_pcs_main_n1_win1.pdf}
        \label{sfig:fgwnlogpcsmain}
    \end{subfigure}
    \begin{subfigure}{0.32\textwidth}
        \caption{Gross Wealth (k€, winsored)}
        \includegraphics[width=.95\linewidth]{../../output/figures/csdid2/h_descr/f_gw_pcs_main_n1_win1.pdf}
        \label{sfig:fgwpcsmain}
    \end{subfigure}
    \begin{subfigure}{0.32\textwidth}
        \caption{Mental Health (oblique)}
        \includegraphics[width=.95\linewidth]{../../output/figures/csdid2/h_descr/f_pcs_main_mcs_main_n1_win1.pdf}
        \label{sfig:fpcsmainmcsmain}
    \end{subfigure}
    \begin{subfigure}{0.32\textwidth}
        \caption{Unemployment exp. (months)}
        \includegraphics[width=.95\linewidth]{../../output/figures/csdid2/h_descr/f_expue_pcs_main_n1_win1.pdf}
        \label{sfig:fexpuepcsmain}
    \end{subfigure}
    \begin{subfigure}{0.32\textwidth}
        \caption{Satisfaction With Health}
        \includegraphics[width=.95\linewidth]{../../output/figures/csdid2/h_descr/f_sats_health_pcs_main_n1_win1.pdf}
        \label{sfig:fsatshealthpcsmain}
    \end{subfigure}
    % start main caption
    \caption{Development of selected variables from event study (Physical Health Domain)}
    \label{fig:inlevelspcsmain}
    \fnote{Panels depict the evolution of selected variables for the
        treated group (in red) and the untreated (in blue). In panel (a), one can see that, on average, the untreated do experience a strong direct impact at $e=0$
        but it recovers to the previous values in the next period ($e=2$), while the treated recovers only
        slightly and carry on on a downward trend. In panel (d), one can see the cross effect on the other health domain. That is,
        how does the shock in the one health dimension affect the other health dimension. Whiskers depict the 99\% confidence interval. \\
        Notes:\\
        - a: The untreated group, by definition, are not
        assigned a ``event year'', so their event is set as the year when they experience their worst
        health outcome, albeit still below the threshold. The actual DiD procedure do not impute any treatment
        year for the untreated, but for this visualization, one has to set a date or assign randomly and any
        choice would have some drawbacks. This choice is able depicting the evolution of people on limit of treatment assignment.\\
        - b: The vertical line depict the last period before (imputed) treatment taking place.\\
        - c: To avoid sample composition affecting the interpretation of these figures, the sample is restricted to a balanced panel.}
\end{figure}

\begin{figure}[htb!]
    % -------------------------------------------------------------------------------------------------
    % var: mcs_main         % Mental Health Domain
    % -------------------------------------------------------------------------------------------------
    % -------------------------------------------------------------------------------------------------
    \centering \setcounter{subfigure}{0}% Reset subfigure counter
    %%A: Mental Health Domain
    %\\ %%%%%%%%%%%%%%%%%%%%%%%%%%%%%%%%%%%%%%%%%%%%%%%%%%%%%%%%%%%%%%%%%%%%%%%%%%%%%%%%%%%%%%%%%%%
    \begin{subfigure}{0.32\textwidth}
        \caption{Mental Health (oblique)}
        \includegraphics[width=.95\linewidth]{../../output/figures/csdid2/h_descr/f_mcs_main_mcs_main_n1_win1.pdf}
        \label{sfig:fmcsmainmcsmain}
    \end{subfigure}
    \begin{subfigure}{0.32\textwidth}
        \caption{Gross Wealth (log)}
        \includegraphics[width=.95\linewidth]{../../output/figures/csdid2/h_descr/f_gw_nlog_mcs_main_n1_win1.pdf}
        \label{sfig:fgwnlogmcsmain}
    \end{subfigure}
    \begin{subfigure}{0.32\textwidth}
        \caption{Gross Wealth (k€, winsored)}
        \includegraphics[width=.95\linewidth]{../../output/figures/csdid2/h_descr/f_gw_mcs_main_n1_win1.pdf}
        \label{sfig:fgwmcsmain}
    \end{subfigure}
    \begin{subfigure}{0.32\textwidth}
        \caption{Physical Health (oblique)}
        \includegraphics[width=.95\linewidth]{../../output/figures/csdid2/h_descr/f_mcs_main_pcs_main_n1_win1.pdf}
        \label{sfig:fmcsmainpcsmain}
    \end{subfigure}
    \begin{subfigure}{0.32\textwidth}
        \caption{Unemployment exp. (months)}
        \includegraphics[width=.95\linewidth]{../../output/figures/csdid2/h_descr/f_expue_mcs_main_n1_win1.pdf}
        \label{sfig:fexpuemcsmain}
    \end{subfigure}
    \begin{subfigure}{0.32\textwidth}
        \caption{Satisfaction With Health}
        \includegraphics[width=.95\linewidth]{../../output/figures/csdid2/h_descr/f_sats_health_mcs_main_n1_win1.pdf}
        \label{sfig:fsatshealthmcsmain}
    \end{subfigure}
    % start main caption
    \caption{Development of selected variables from event study (Mental Health Domain)}
    \label{fig:inlevelsmcsmain}
    \fnote{Notes: points a: b: and c: of \cref{fig:inlevelspcsmain} also apply, as well as similar interpretation of each panels.
        Emphasis here to panel (e), showing a nice example of the effect taking place after not only parallel but equal pre-trend.}
\end{figure}



\begin{figure}
    \centering
    {Kaplan--Meier survival estimates}\par
    \includegraphics[width=0.7\linewidth]{../../output/figures/attrition/kpme_surv_nw_qile_age}
    \caption{Survival analysis of SOEP participants by wealth quintiles}
    \label{fig:kpmesurvnwqileage}
    \fnote{
        Notes: This graph evidentiates the gradient in panel attrition by wealth levels.
        It depicts an unadjusted survival estimate of participation length for each quintile of 
        age-adjusted gross wealth.  While the four upper quintile groups show a similar survival rate, 
        those in the bottom quintile are less likely to remain as long in the SOEP.
        The graph suggests that the differential attrition 
        rate is stronger in the first 5 years. From then on, the curves evolve parallel to one another. 
        For this analysis, SOEP's entering yeard is used, including if it happened before 2002. Those that dropped out
        before 2002 (from when on wealth data are available) are not considered for this examination. 
        Data source: SOEPv37.
        }
\end{figure}






% -------------------------------------------------------------------------------------------------
% -------------------------------------------------------------------------------------------------
% -------------------------------------------------------------------------------------------------
% -------------------------------------------------------------------------------------------------
% PCS_DEF ----
\chapter{Replications based on the SF-12 method}%
%
% -------------------------------------------------------------------------------------------------
% -------------------------------------------------------------------------------------------------
% -------------------------------------------------------------------------------------------------
\begin{figure}
    % -------------------------------------------------------------------------------------------------
    % var: pcs_def         % Physical Health Domain
    % -------------------------------------------------------------------------------------------------
    % -------------------------------------------------------------------------------------------------
    \centering \setcounter{subfigure}{0}% Reset subfigure counter
    \renewcommand{\thesubfigure}{P\alph{subfigure}}
    Physical Health Domain
    \\ %%%%%%%%%%%%%%%%%%%%%%%%%%%%%%%%%%%%%%%%%%%%%%%%%%%%%%%%%%%%%%%%%%%%%%%%%%%%%%%%%%%%%%%%%%%
    \begin{subfigure}{0.32\textwidth}
        \caption{Physical Health (oblique)}
        \includegraphics[width=.95\linewidth]{../../output/figures/csdid2/h_descr/f_pcs_def_pcs_def_n1_win1.pdf}
        \label{sfig:fpcsdefpcsdef}
    \end{subfigure}
    \begin{subfigure}{0.32\textwidth}
        \caption{Gross Wealth (log)}
        \includegraphics[width=.95\linewidth]{../../output/figures/csdid2/h_descr/f_gw_nlog_pcs_def_n1_win1.pdf}
        \label{sfig:fgwnlogpcsdef}
    \end{subfigure}
    \begin{subfigure}{0.32\textwidth}
        \caption{Gross Wealth (k€, winsored)}
        \includegraphics[width=.95\linewidth]{../../output/figures/csdid2/h_descr/f_gw_pcs_def_n1_win1.pdf}
        \label{sfig:fgwpcsdef}
    \end{subfigure}
    \begin{subfigure}{0.32\textwidth}
        \caption{Mental Health (oblique)}
        \includegraphics[width=.95\linewidth]{../../output/figures/csdid2/h_descr/f_pcs_def_mcs_def_n1_win1.pdf}
        \label{sfig:fpcsdefmcsdef}
    \end{subfigure}
    \begin{subfigure}{0.32\textwidth}
        \caption{Unemployment exp. (months)}
        \includegraphics[width=.95\linewidth]{../../output/figures/csdid2/h_descr/f_expue_pcs_def_n1_win1.pdf}
        \label{sfig:fexpuepcsdef}
    \end{subfigure}
    \begin{subfigure}{0.32\textwidth}
        \caption{Satisfaction With Health}
        \includegraphics[width=.95\linewidth]{../../output/figures/csdid2/h_descr/f_sats_health_pcs_def_n1_win1.pdf}
        \label{sfig:fsatshealthpcsdef}
    \end{subfigure}
%    % start def caption
%    \caption{Development of selected variables from event study (Physical Health Domain)}
%    \label{fig:inlevelspcsdef}
%    \fnote{}
%\end{figure}
%\begin{figure}[htb!]
%    % -------------------------------------------------------------------------------------------------
%    % var: mcs_def         % Mental Health Domain
%    % -------------------------------------------------------------------------------------------------
%    % -------------------------------------------------------------------------------------------------
    \centering \setcounter{subfigure}{0}% Reset subfigure counter
    \renewcommand{\thesubfigure}{M\alph{subfigure}}
    Mental Health Domain
    \\ %%%%%%%%%%%%%%%%%%%%%%%%%%%%%%%%%%%%%%%%%%%%%%%%%%%%%%%%%%%%%%%%%%%%%%%%%%%%%%%%%%%%%%%%%%%
    \begin{subfigure}{0.32\textwidth}
        \caption{Mental Health (oblique)}
        \includegraphics[width=.95\linewidth]{../../output/figures/csdid2/h_descr/f_mcs_def_mcs_def_n1_win1.pdf}
        \label{sfig:fmcsdefmcsdef}
    \end{subfigure}
    \begin{subfigure}{0.32\textwidth}
        \caption{Gross Wealth (log)}
        \includegraphics[width=.95\linewidth]{../../output/figures/csdid2/h_descr/f_gw_nlog_mcs_def_n1_win1.pdf}
        \label{sfig:fgwnlogmcsdef}
    \end{subfigure}
    \begin{subfigure}{0.32\textwidth}
        \caption{Gross Wealth (k€, winsored)}
        \includegraphics[width=.95\linewidth]{../../output/figures/csdid2/h_descr/f_gw_mcs_def_n1_win1.pdf}
        \label{sfig:fgwmcsdef}
    \end{subfigure}
    \begin{subfigure}{0.32\textwidth}
        \caption{Physical Health (oblique)}
        \includegraphics[width=.95\linewidth]{../../output/figures/csdid2/h_descr/f_mcs_def_pcs_def_n1_win1.pdf}
        \label{sfig:fmcsdefpcsdef}
    \end{subfigure}
    \begin{subfigure}{0.32\textwidth}
        \caption{Unemployment exp. (months)}
        \includegraphics[width=.95\linewidth]{../../output/figures/csdid2/h_descr/f_expue_mcs_def_n1_win1.pdf}
        \label{sfig:fexpuemcsdef}
    \end{subfigure}
    \begin{subfigure}{0.32\textwidth}
        \caption{Satisfaction With Health}
        \includegraphics[width=.95\linewidth]{../../output/figures/csdid2/h_descr/f_sats_health_mcs_def_n1_win1.pdf}
        \label{sfig:fsatshealthmcsdef}
    \end{subfigure}
    % start main caption
    \caption{Development of selected variables based on SF-12 method}
    \label{fig:inlevelsmcsdef}
    \fnote{Notes: Replication of \cref{fig:inlevelspcsmain,fig:inlevelsmcsmain} using PCS and MCS following the SF-12 method.
    Here one can observe the negative cross-effect in \subref{sfig:fpcsdefmcsdef} and \subref{sfig:fmcsdefpcsdef} where the adverse outcome 
    in one dimension impacts the other dimension in the opposite direction. The pre-treatment trend, specially in \subref{sfig:fexpuepcsdef} 
    and \subref{sfig:fexpuemcsdef}, are less supportive of the parallel trends assumption than in the alternative method.
    Same considerations regarding the ``treatment date'' of the untreated stated in the notes of  \cref{fig:inlevelspcsmain} apply here as well.}
\end{figure}





% ---------------------------------------------------------------------------------------------------------------------
% Validation with objective health diagnoses
% ---------------------------------------------------------------------------------------------------------------------
\begin{figure}[tb!]
    \centering
    \begin{subfigure}{0.45\textwidth}
        \caption{Depression}
        \includegraphics[width=.95\linewidth]{../../output/figures/csdid2/c_othervars/f_2_depres_ev_1o2_def.pdf}
    \end{subfigure}
    \begin{subfigure}{0.45\textwidth}
        \caption{Sleep Disorder}
        \includegraphics[width=.95\linewidth]{../../output/figures/csdid2/c_othervars/f_2_sleep_ev_1o2_def.pdf}
    \end{subfigure}
    \caption{Replication of validation of mental health diagnoses with orthogonal health scores} 
    \label{fig:csdid_hltdiag_rep}
    \fnote{Notes: This replication aims to show that also using orthogonal health scores, the effect follows 
        a similar in both domains, similar to using oblique scores. 
        Response variables are binary indicators of being ever diagnosed with the respective condition.
        Panels have different scale on the y-axis. Whiskers depict the 95\% confidence interval}
\end{figure}








































% -------------------------------------------------------------------------------------------------
% figure: SF-12 default pcs mcs
% -------------------------------------------------------------------------------------------------
\begin{figure}[ht!]
    \centering
    \begin{subfigure}{0.45\textwidth}
        \caption{Gross Wealth (log)\\(M1\tsub{sf12} / P1\tsub{sf12})}
        \includegraphics[width=.95\linewidth]{../../output/figures/csdid2/b_mcspcs/f_12_gw_nlog_1o2_ct2.pdf}
        \label{sfig:did_sf12_a}
    \end{subfigure}
    \begin{subfigure}{0.45\textwidth}
        \caption{Net Wealth (neglog)\\(M2\tsub{sf12} / P2\tsub{sf12})}
        \includegraphics[width=.95\linewidth]{../../output/figures/csdid2/b_mcspcs/f_22_nw_nlog_1o2_ct2.pdf}
        \label{sfig:did_sf12_b}
    \end{subfigure}
    \begin{subfigure}{0.45\textwidth}
        \caption{Gross Wealth (k€, winsored)\\(M3\tsub{sf12} / P3\tsub{sf12})}
        \includegraphics[width=.95\linewidth]{../../output/figures/csdid2/b_mcspcs/f_32_gw_1o2_ct2.pdf}
        \label{sfig:did_sf12_c}
    \end{subfigure}
    \begin{subfigure}{0.45\textwidth}
        \caption{Net Wealth (k€, winsored)\\(M4\tsub{sf12} / P4\tsub{sf12})}
        \includegraphics[width=.95\linewidth]{../../output/figures/csdid2/b_mcspcs/f_42_nw_1o2_ct2.pdf}
        \label{sfig:did_sf12_d}
    \end{subfigure}
    \caption{Replication of main analysis with SF-12 scores}
    \label{fig:did_sf12}
    \fnote{Notes: These panels are a visual representation of the table of coefficients presented in \cref{tab:coefs_sf12}. 
        Post-treatment, the effects are in line with those obtained in the main analysis. Pre-treatment, in contrast, seems 
        to be less supportive of the parallel trends assumptions, including in the primary specifications with log-transformed 
        gross wealth (P1\tsub{sf12} and M1\tsub{sf12}). To what extend this is driven by the particular treatment rule or
        whether an artifact from the sf12 or from the alternative scores computes could be a matter of further inquiry.
        To what extend should be, a priori, expected that the trends remain parallel 8 or 10 years prior to treatment 
        is also worth considering. 
    Whiskers depict the 95\% confidence interval.}
\end{figure}



\begin{table}[htbp!]
    \centering
    \begin{adjustbox}{max width=\textwidth}
        \begin{threeparttable}       
            \caption{Table of coefficients of models based on the SF-12 methodology}
            \label{tab:coefs_sf12} % \setlength{\tabcolsep}{4pt}
            \begin{tabular}{l*4{P}*4{M}}
                \toprule
                & \multicolumn{4}{l}{Physical Health}   & \multicolumn{4}{l}{Mental Health}     \\ \cmidrule(lr{1em}){2-5} \cmidrule(lr{1em}){6-9}
                & \multicolumn{2}{l}{(neg)log}               & \multicolumn{2}{l}{level} & \multicolumn{2}{l}{(neg)log} & \multicolumn{2}{l}{level} \\ \cmidrule(lr{1em}){2-3} \cmidrule(lr{1em}){4-5} \cmidrule(lr{1em}){6-7} \cmidrule(lr{1em}){8-9} 
                & {gross (\%)} & {net (\%)$^1$} & {gross} & {net} & {gross (\%)} & {net (\%)$^1$} & {gross} & {net}               \\
                & {(P1\tsub{sf12})}  & {(P2\tsub{sf12})}            & {(P3\tsub{sf12})}  & {(P4\tsub{sf12})}              & {(M1\tsub{sf12})}  & {(M2\tsub{sf12})}            & {(M3\tsub{sf12})}  & {(M4\tsub{sf12})}              \\  
                \midrule
                \input{./incl/tbls/tbl_full_def_bd_1o2_ct2.tex}% &
            \end{tabular}%
            \begin{tablenotes}[para,flushleft]%
                \vspace*{-\baselineskip}
                {\raggedleft*~$p<0.05$,~**~$p<0.01$,~***~$p<0.001$\\}
                Notes: This results are a replication of \cref{tab:main_res_event} based on the SF-12 methodology for construction PCS and MCS. 
                Wild Bootstrap standard error in parenthesis. 
                A visual representation of this results are depicted in \cref{fig:did_sf12}
                $^1$Coefficients (and standard errors) of log specifications are transformed to represent the effect in percentage terms, but
                such interpretation of the neglog transformation might be biased (see \cref{sec:transformcoefs})
            \end{tablenotes}
        \end{threeparttable}
    \end{adjustbox}
\end{table}
% -------------------------------------------------------------------------------------------------
% end table 
% -------------------------------------------------------------------------------------------------



