\newpage
\appendix
\lstset{literate=
	{á}{{\'a}}1 {é}{{\'e}}1 {í}{{\'i}}1 {ó}{{\'o}}1 {ú}{{\'u}}1
	{Á}{{\'A}}1 {É}{{\'E}}1 {Í}{{\'I}}1 {Ó}{{\'O}}1 {Ú}{{\'U}}1
	{à}{{\`a}}1 {è}{{\`e}}1 {ì}{{\`i}}1 {ò}{{\`o}}1 {ù}{{\`u}}1
	{À}{{\`A}}1 {È}{{\'E}}1 {Ì}{{\`I}}1 {Ò}{{\`O}}1 {Ù}{{\`U}}1
	{ä}{{\"a}}1 {ë}{{\"e}}1 {ï}{{\"i}}1 {ö}{{\"o}}1 {ü}{{\"u}}1
	{Ä}{{\"A}}1 {Ë}{{\"E}}1 {Ï}{{\"I}}1 {Ö}{{\"O}}1 {Ü}{{\"U}}1
	{â}{{\^a}}1 {ê}{{\^e}}1 {î}{{\^i}}1 {ô}{{\^o}}1 {û}{{\^u}}1
	{Â}{{\^A}}1 {Ê}{{\^E}}1 {Î}{{\^I}}1 {Ô}{{\^O}}1 {Û}{{\^U}}1
	{œ}{{\oe}}1 {Œ}{{\OE}}1 {æ}{{\ae}}1 {Æ}{{\AE}}1 {ß}{{\ss}}1
	{ű}{{\H{u}}}1 {Ű}{{\H{U}}}1 {ő}{{\H{o}}}1 {Ő}{{\H{O}}}1
	{ç}{{\c c}}1 {Ç}{{\c C}}1 {ø}{{\o}}1 {å}{{\r a}}1 {Å}{{\r A}}1
	{€}{{\euro}}1 {£}{{\pounds}}1 {«}{{\guillemotleft}}1
	{»}{{\guillemotright}}1 {ñ}{{\~n}}1 {Ñ}{{\~N}}1 {¿}{{?`}}1,
	basicstyle=\footnotesize\ttfamily,
	numbers=left,
	numbersep=5pt,
	numberstyle=\scriptsize, 
	showspaces=false,               % show spaces adding particular underscores 
	showstringspaces=false,         % underline spaces within strings 
	tabsize=4,
	breaklines=true,                % sets automatic line breaking 
	frame=single,                   % adds a frame around the code
}


% appendix start here  % appendix start here  % appendix start here  
\chapter{Appendix}
% appendix start here  % appendix start here  % appendix start here  

\section{WMS server implementation and comparison}
\label{app:wms}


For the results shown in \cref{fig:wms_all} under \cref{sec:wms} the following URL and layer settings were used: 


\begin{enumerate}[label=(\alph*)]\setlength{\itemsep}{-5pt}
	\item Aerial Image from 2015 (FIS-Broker) \\
	URL: \url{http://fbinter.stadt-berlin.de/fb/wms/senstadt/k_luftbild2015_rgb} \\
	Layer: 0
	
	\item Colour OpenStreetMap (terrestris) \\
	URL: \url{http://ows.terrestris.de/osm/service} \\
	Layer: OSM-WMS
	
	\item Aerial Image from 1928 (FIS-Broker)\\
	URL: \url{http://fbinter.stadt-berlin.de/fb/wms/senstadt/k_luftbild1928} \\
	Layer: 0
	
	\item Grayscale OpenstreetMap (terrestris)\\
	URL: \url{http://ows.terrestris.de/osm-gray/service}\\
	Layer: OSM-WMS
	
	\item Berlin Map from 1940 (FIS-Broker)\\
	URL: \url{http://fbinter.stadt-berlin.de/fb/wms/senstadt/berlin1940}\\
	Layer: 0\\
	Notes: This map covers only relatively small area over the centre of Berlin
	
	\item Built-in OpenStreetMaps (JMP)\\
	URL and layer not required.
\end{enumerate}

\noindent Note also that, in order to better emphasise the data points, the marker colour was changed appropriately and the transparency level set to 0.7 on map images (Figs. b), d), e) and f)). Such changes can be done via the \textit{Customize...} window [on Graph frame: Right-click > Customize...].




\section{R scripts geocoding Stolpersteine addresses}
\label{sec:appendix_geocode}

For this code implementation I separated the process into three parts. The first part deals with the preparation of the data, including making sure there are not repeated addresses with slightly different names, such as "ss" and "ß" or even lower and upper case. Additionally, given that multiple observations refer to the exact same address, a column with a unique identifier for each address was added, so that each observation needed to be geocoded only once.  In the end, a smaller csv file containing only the unique address and their index number is saved for the geocoding process in the second step.

\begin{lstlisting}[title=R code to geocode addresses (1), frame=single]
# file format: UTF-8

# read csv file 
df <- read.csv("./input/stolper_07.09.17.csv",
			   header = T, sep = ",", encoding = "UTF-8",
			   strip.white = TRUE, na.strings = "",
    		   stringsAsFactors=FALSE)

# rename column 3 from "District." to "District"
# and add missing distric name
colnames(df)[3] = "District"                          
df$District[is.na(df$District)] <- "Schmargendorf"

# make new column with address and ortsteil
df$address.full <- paste(df$Address, df$District,
						 "Berlin, Deutschland", sep=" - ")

# change all to lower case and umlauts to latin characters.
df$address.full <- tolower(df$address.full)
df$address.full <- sub("str\\.", "strasse", df$address.full)
df$address.full <- sub("pl\\.", "platz", df$address.full)
df$address.full <- sub("ü", "ue", df$address.full)
df$address.full <- sub("ö", "oe", df$address.full)
df$address.full <- sub("ä", "ae", df$address.full)
df$address.full <- sub("é", "e", df$address.full)
df$address.full <- sub("è", "e", df$address.full)
df$address.full <- sub("ß", "ss", df$address.full)

# sort by address 
df <- df[order(df$address.full), ]

# Add unique index
df <- transform(df, ClusterID = as.numeric(
				interaction(df$address.full, drop=TRUE)))
		
max(df$ClusterID) # = 2843 unique addresses
# write ongoing output into csv file
write.csv(df, file = "./input/dfCleaned.csv",
		  fileEncoding = "UTF-8", row.names = FALSE)

# make a small data frame with ID and full address only
dfSmall <- unique(data.frame("ClusterID" = df$ClusterID,
					 		 "Address" =  df$address.full))

# save preparation data into csv file
write.csv(dfSmall, file="./input/dfSmall.csv", fileEncoding="UTF-8")
\end{lstlisting}


In the second step is where the geocoding process takes place. The following lines of code are a little more complex and surely not in the most sophisticated way\footnote{A better script for a geocoding a large list of addressees, which was helpful when creating my own, can be found on: \url{https://www.shanelynn.ie/massive-geocoding-with-r-and-google-maps}, Accessed: 03.10.2017.}, but it accomplishes the intended results. Some complexity is added, due to the fact that Google's API user conditions sets a daily query limit at 2500. With that in mind, I created a way of saving the ongoing results for retrieval the following day or if there should be an internet connection problem.  

\begin{lstlisting}[title=R code to geocode addresses (2)]
library(ggmap)

# set location for ongoing-results file
ongoing_results_file <- "./output/ongoing_results.rds"

# open ongoing if exists, otherwise open dfSmall from preparation 
if (file.exists(ongoing_results_file)) {
	dfSmall <- readRDS(ongoing_results_file)
	startIndex <- nrow(dfSmall) - sum(is.na(dfSmall$lon))
	print("using ongoing file as dfSmall")
} else {
	dfSmall <- read.table("./input/dfSmall.csv",
	header=T, sep=",", encoding="UTF-8",
	colClasses=c('NULL',NA,NA), stringsAsFactors=FALSE)
	
	# create new column for lon, lat and address to be later populated
	dfSmall$lon <- NA 
	dfSmall$lat <- NA  
	dfSmall$address <- NA
	startIndex <- 1
	print("using dfSmall from preparation step (1)")
}

# function to go through the rows and geocode each address
for(i in seq(startIndex, nrow(dfSmall))) {
	id <- dfSmall$ClusterID[i]
	
	if (is.na(dfSmall$lon[i])) {
		#checks if row is a "missing value"
		address <- dfSmall$Address[i]
		
		# call geocode function and apply lat, lon to cluster ID rows
		dfAPIResponse <- geocode(address, output="latlona", override_limit=T)
		dfAPIResponse$ClusterID[1] <- id
			dfAPIResponse$FullAddress <- dfSmall$Address[i]
		dfSmall$lat[dfSmall$ClusterID == id] <- 
			dfAPIResponse$lat[dfAPIResponse$ClusterID == id]
		dfSmall$lon[dfSmall$ClusterID == id] <- 
			dfAPIResponse$lon[dfAPIResponse$ClusterID == id]
			
		# when retrieval is not successfull the column "address"
		# is not filled, therefore the next if-else statement
		if( "address" %in% names(dfAPIResponse) ) {
		dfSmall$address[dfSmall$ClusterID == id] <- 
			dfAPIResponse$address[dfAPIResponse$ClusterID == id]
		} else {
			dfSmall$address[dfSmall$ClusterID == id] <- NA
		}
	
		# save ongoing results every 10th query
		if (i %% 10 == 0 | i == nrow(dfSmall)) {
		saveRDS(dfSmall, file = ongoing_results_file)  
		}
	} else {
	print(paste("id nr", id, "already geocoded"))
	}
}
# logging in case of errors 
w <- warnings()
sink("./log/warnings.txt")
print(w)
sink()

# write final results into csv file 
write.table(dfSmall, file = "./output/dfResults.csv", sep = ",",
			fileEncoding = "UTF-8",row.names = FALSE)
\end{lstlisting}

Finally, the geocoded addresses can be merged back into the main data frame in the third step. The csv file created is then used for further analysis within JMP.

\begin{lstlisting}[title=R code to geocode addresses (3)]
# read results data
dfResults <- read.table("./output/dfResults.csv",
						header=T, sep=",",
						encoding="UTF-8",
						stringsAsFactors=FALSE)

# read full input data (already cleaned)
df <- read.table("./input/dfCleaned.csv",
				 header = T, sep = ",", encoding = "UTF-8",
				 na.strings = "", stringsAsFactors=FALSE)

# join both data.frames by matching "ClusterID"
dfTotal <- join(df, dfResults, type = "full", by = "ClusterID")

# after checking that the address matches, drop duplicate columns
list_keep <- c("Name", "Address", "District",
			   "Year", "ClusterID", "lon", "lat")
			   
dfTotal <- dfTotal[ , (names(dfTotal) %in% list_keep)]

# rename some columns for better consistency among oder headers
names(dfTotal)[6] <- "Longitude"; names(dfTotal)[7] <- "Latitude"

# write final results into csv file for analysis in JMP
write.table(dfTotal, file = "./output/stolpersteine.csv",
			fileEncoding = "UTF-8", sep = ",", row.names = FALSE)
\end{lstlisting}


\section{R scripts for shapefile coordinate system conversion}
\label{app:shapefile_conv}



\begin{lstlisting}[title=R code to convert shapefile from UTM33 N to WGS84]
# ref: https://stackoverflow.com/questions/36520915/
# https://gis.stackexchange.com/questions/166876/

# libraries and working directory
library(rgdal)

# CRS for input and output formats
epsg_input = "+init=epsg:25833" # change input CRS accordingly
epsg_output = "+init=epsg:4326" # change output CRS accordingly

# read ESRI shapefile 
Berlinshape <- readOGR("./inputDATA/LOR_BERLIN_ALL/LOR_SHP_EPSG_25833_UTM33/.",
					   "Planungsraum_EPSG_25833")

# assign UTM33 projection via proj4strig
proj4string(Berlinshape) <- CRS(epsg_input)

# convert into WGS84 Coordinate System
Berlinshape <- spTransform(Berlinshape, CRS(epsg_output))

#write new shapefile on WGS84 Coordinate System 
writeOGR(Berlinshape,
		"./outputDATA/WGS84/from_SHP_UTM33",   # file location
		"BERLIN_LOR_PLR_WGS84",    # file prefix
		driver = "ESRI Shapefile")
\end{lstlisting}


\begin{lstlisting}[title=R code for aggregating Stolpersteine in LOR boundaries]
# helpful resource:
# https://gis.stackexchange.com/questions/110117/

# read ESRI shapefile downloaded from Berlin OPENDATA portal 
BerlinShape <- readOGR("./outputDATA/WGS84/from_SHP_UTM33/.",
					   "BERLIN_LOR_PLR_WGS84")

# read stolpersteine csv as data.frame
stolpersteine <- read.table("./stolpersteine/output/stolpersteine.csv",
							header=T, sep=",",
							encoding="UTF-8",
							stringsAsFactors=FALSE)

# drop NA's from data.frame
stolpersteine <- stolpersteine[!is.na(stolpersteine$Latitude),]

# set Longintude and Latitude columns as coordinates, this 
# also turns the data.frame into a SpatialPolygonsDataFrame object
coordinates(stolpersteine) <- ~ Longitude + Latitude

# check proj4string of each object
proj4string(stolpersteine) ;proj4string(BerlinShape)

# assign WGS84 CS via proj4string
proj4string(stolpersteine) <- proj4string(BerlinShape)

# creates data frame with Stolpersteien points
# contained in Berlinshape
df_aggregate <- over(stolpersteine, BerlinShape)

# aggregate results and turn into data.frame class
df_aggregate <- data.frame(table(df_aggregate$SCHLUESSEL))

# rename columns
names(df_aggregate) <- c("LOR_PLR", "Stolpersteine")

# set Bezirks- und Prognosenräume columns.
df_aggregate$LOR_BZR <- substr(df_aggregate$LOR_PLR, 1, 6)
df_aggregate$LOR_PRR <- substr(df_aggregate$LOR_PLR, 1, 4)
# add a district feature (technically not a LOR)
df_aggregate$District <- substr(df_aggregate$LOR_PLR, 1, 2)

# reorder columns 
df_aggregate <- df_aggregate[c(5,4,3,1,2)]

# write into csv 
write.table(df_aggregate, file = "./stolpersteine/output/stolper_choropleth.csv",
			fileEncoding = "UTF-8", sep = ",", row.names = FALSE)

\end{lstlisting}

\section{R scripts for retrieving centroid from LOR layers }
\label{app:r_script_centroid}

The following script was used to retrieve the centroids from each LOR layer and aggregating the number of Stolpersteine inside each cell. This example goes over the PLR layer, but the same procedure was done by adjusting the layer naming accordingly. 

\begin{lstlisting}[title = R script for retrieving centroids and aggregating on LOR layer]
# import libraries
library(rgdal)
library(sp)

# read LOR_PLR from WGS84 shapefile
BerlinShapePLR <- readOGR(dsn = "./outputDATA/WGS84/from_SHP_UTM33/.",
                          layer = "BERLIN_LOR_PLR_WGS84")

# get centroid coordinates and IDs of shapefile
centroidsPLR <- as.data.frame(coordinates(BerlinShapePLR))
LOR_PLR <- as.data.frame(BerlinShapePLR$SCHLUESSEL)

# create a data.frame with IDS and centroids and change column names
berlinPLRCentr <- cbind(LOR_PLR, centroidsPLR)
names(berlinPLRCentr) <- c("LOR_PLR", "Longitude", "Latitude")

# read stolpersteine csv file
fileName <- "./stolpersteine/output/stolper_choropleth.csv"
berlinStolper <- read.csv(file = fileName, 
colClasses = c(rep("character", 4), "numeric"))

# aggregate (sum) over Stolpersteine by common LOR_PLR
berlinAggPLR <- aggregate(berlinStolper$Stolpersteine,
						  by = list(berlinStolper$LOR_PLR),
						  FUN = sum)

# apply name (important LOR_PLR matches for merge)
names(berlinAggPLR) <- c("LOR_PLR", "Stolpersteine")

# merge and write to file as csv 
berlinAggPLR <- base::merge(berlinPLRCentr, berlinAggPLR)
write.table(berlinAggPLR,
		    file = "./stolpersteine/output/berlinStolpLORPLR.csv",
		    fileEncoding = "UTF-8", sep = ",", row.names = FALSE)
\end{lstlisting}