
In this section, revisiting the theoretical groundwork explored in \cref{sec:overview_did}, I
present the empirical methodology applied to the current setting.

\section{Event Study Framework}
\label{sec:empir_event_study}

In the main analysis, the aim is to estimate the dynamics of wealth accumulation which can be
attributed to a significant health degradation in the physical and mental domain. That is, how does
experiencing a negative health outcome affects wealth accumulation over subsequent years in a
relatively long term. With this in mind, the parameter of interest is\footnote{This might be an
    unfamiliar presentation on the event study framework, for an analogous version based on a regression-based design,
    refer to \cref{app:twfe}.} %
\begin{equation}
    \theta_\text{es}(e) = \sum_{g \in \mathcal{G}} \mathbb{1}\big[ g + e \leq \mathcal{T} \big] P(G=g|G+e \leq \mathcal{T}) ATT(g,g+e), 
    \tag{\ref{eq:theta_es} revisited}
    \label{eq:theta_es2}
\end{equation}
%
where $e = t-g$, $t \in \{2002, 2004, ..., 2020\}$,  $g \in \mathcal{G}$, where $\mathcal{G} = \{2004, 2006,
..., 2020\}$ and $\mathcal{T}=2020$. Note that the first item of $\mathcal{G}$ is 2004, because always-treated
units are dropped from the analysis. Additionally, people assigned to treatment in their first survey
year are also not considered in the analysis.

The relative time window is restricted to $e=\{-10,-8,...,10,12\}$ years, as estimates outside this range
become very imprecise. The time of treatment, i.e., when an individual first experiences a negative health
outcome happens between $e=-2$ and $e=0$, but is only becomes observable in the data when $e=0$. The base or
reference period, is set to $e=-2$, which is the last period before treatment.


\subsection{Treatment Assignment Rule}
\label{sec:treatassigrule}

Everything in what follows is conducted twice: once in relation to the physical domain and once in relation to
the mental domain. For brevity, however, I subsequently describe the rule using a general term
\textit{health outcome}.

An individual $i$ is assigned to group $g=t$ if they experience a relatively severe adverse health outcome
compared to people of the same age and gender, as measured by the respective component summary score (PCS
for the physical and MCS for the mental domain). Specifically, if the individual's summary score is half
a standard deviation worse than the median score of people of the same age and gender, they become a candidate for
treatment assignment.\footnote{Alternative threshold and rules are tested in robustness checks.} If
this individual experiences a negative health outcome at least once more (though not necessarily
consecutively), they are indeed assigned to treatment in the period when they first experience the adverse
health outcome. This decision has two primary reasons: first, to prevent treatment assignment due to a small
fluctuation around the threshold, and second, by employing a less strict assignment rule, approximately half of
the sample would eventually be assigned to treatment. Thus, by restricting to those experiencing it twice, the
analysis focuses on individuals with a more severe health deterioration. Further, the age--gender adjustment, 
instead of using the simple difference was done to account for gender--age specific reporting heterogeneity, as
indicated by \textcite{ziebarth2010measurement}. 


The threshold of one-half standard deviation aligns with the findings of
\textcite[568]{vilagut.etal2013mental}, who identify 45.6 as the optimal cutoff point for $\text{MCS}_{sf12}$
to evaluate 30-day depressive disorders. Given that our health components have a mean of 50 and a standard
deviation of 10, half a standard deviation is approximately equal to the identified cutoff point. On the other
hand, the comparison (or control) group consists of people with more stable health trajectories, experiencing a
health outcome worse than the threshold at most once.

Note that, for clarity, I also use the term ``shock'' throughout this work to denote the adverse event.
However, it must be emphasized that not all treatment assignments occur due to a drastic score decrease from
one period to the next for a given person. A continuous but slow degradation of their health also characterizes
treatment.  Further, also regarding terminology, I use conventional terms in the causal inference literature
borrowed from the medical domain and since this analysis deals with health outcomes, this can lead to confusion.
To clarify, ``treatment assignment'' here simply means that the individual experiences a bad health outcome,
without taking into account any medical treatments, in the usual common sense of the world.



\subsection{Wealth Measures}

For this analysis, I use four measures of wealth: gross and net wealth, as well as these measured in
levels and in logs. The variables in levels are winsored at the 1st and 99th percentile. In the case of 
net wealth, the \textit{neglog} transformation was applied, where%
%
\begin{equation}
    W^{\text{neglog}} = \sign(W) \cdot \log{ ( 1 + \abs{ W } ) }.
    \label{eq:neglog}
\end{equation}

The neglog transformation possesses several interesting features, beyond its primary goal of reducing the
skewness of the input data. Notably, it retains the value of zero due to the $\text{sign}$ function. Moreover,
for large positive values of $W$, it behaves as $\log(W)$, and for large negative values, it behaves like
$-\log(W)$. For values around 0, it approximates $W$ linearly. The same transformation is applied to gross
wealth, but as it is non-negative, the transformation simplifies to $\log{(1+W)}$. Additionally, winsoring the
variables at the 1st and 99th percentiles has been shown to enhance the stability of the models.

\subsubsection{On the parallel trends assumption} Crucial to effect identification is the assumption of
Parallel Trends (PT). Dealing with health accumulation over time, it's essential to critically consider what
this assumption entails. In principle, one could question whether trends are more likely to be parallel when
dealing with the wealth variable in absolute or relative terms. This boils down to whether an additive or a
multiplicative wealth accumulation process better models the wealth trajectory over the life cycle. Hence, I
consider wealth in levels and in logs, but a priori, find it  plausible that for different starting values of
wealth, the multiplicative process is likely to capture the accumulation path of different groups better.
Concretely, it seems more credible that richer and poorer people can accumulate a similar percentage of their
current wealth, rather than expecting their wealth to increase by values in absolute term.


\subsubsection{On wealth measure and transformation choice} Both net and gross measures offer compelling
reasons for being the key variable of interest. Net wealth might better capture effective changes in the wealth
trajectory. For instance, if an individual borrows money to buy a house, this would increase gross wealth by
the value of the house but leave net wealth intact until they start paying off the loan. Moreover, experiencing
negative net wealth might be an intriguing aspect in this analysis, given its potential linkages with stress
and anxiety.

On the other hand, the data reveal a considerable number of people with large negative net wealth. Consider an
individual who takes a substantial loan to open a business or another who incurs student loans. If the value of
their assets when opening the business or by the end of their studies is considerably smaller than the value of
their acquired debt, they would be located at the very low end of the net wealth distribution. However, one
would hardly argue that they are the poorest people in the population. On this account, I focus on gross 
wealth as the main measure to capture the socio-economic status of the individual. 


Furthermore, due to the crossing over 0, the interpretation of coefficients from the neglog-transformed models
is not trivial and cannot be simply taken as percentage or proportional effects. While the neglog
transformation might be interesting for modeling and technical reasons, interpretation and economic
implications based on those results are harder to convey. With this reasoning, I consider the model with
log-transformed gross wealth to be the main model in this analysis and focus on those results. However, for
completeness, all results are presented.

